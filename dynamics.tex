\begin{frame}{Complex dynamic systems}
  \bigskip
  \begin{itemize}
  \item<1-> \structure{Objective}
    \par\medskip
    How to develop robust and scalable \alt<2->{\alert<2-3>{\textbf<2-3>{KRR\only<2-3>{\footnote{Knowledge Representation and Reasoning}}}}}{AI} technology
    \\
    for dealing with complex dynamic application scenarios?
    \medskip
  \item<3-> \structure{Approach} via \alert<2-3>{\textbf<2-3>{Answer Set Programming}}
    \medskip
    \begin{enumerate}\normalsize
    \item<4-> \structure{\hyperlink{foundations}{Foundations}} % \ \hyperlink<8>{foundations}{\beamerbutton}
      \only<5>{\par\smallskip\small
        Extend the base logic of ASP, namely the \alert{logic of Here-and-There} (HT), with language elements from
        \begin{itemize}\small
        \item Linear Temporal Logic (LTL)
        \item Linear Dynamic Logic (LDL)
        \end{itemize}
        over a common semantic structure, namely, (finite) HT traces}
    \item<4-> \structure{\hyperlink{telingo}{Design and implementation}} % \ \hyperlink<8>{telingo}{\beamerbutton}
      \only<6>{\par\smallskip
      \begin{itemize}\small
      \item Exploit temporal links
      \item Dedicated temporal reasoning modes
        \smallskip
      \item \url{https://github.com/potassco/telingo}
      \end{itemize}}
    \item<4-8> \structure{\hyperlink{asprilo}{Application}} % \ \hyperlink<8>{asprilo}{\beamerbutton}
      \only<7>{\par\smallskip
        \begin{itemize}\small
        \item Robotic intra-logistics
        \item \url{https://potassco.org/asprilo}
          \smallskip
        \item Formal software engineering
        \item \url{https://github.com/potassco/anthem}
        \end{itemize}}
    \end{enumerate}
  \end{itemize}
\end{frame}
% ----------------------------------------------------------------------
\subsection{Foundations}
% ----------------------------------------------------------------------
\begin{frame}[label=foundations]{The logic of Here-and-There}
  \begin{itemize}
  \item<1->\structure{Origin}%\only<8>{ (\alert<8>{\textbf{monotonic}})}
    \medskip

    Three valued logic due to  (Heyting, 1930; Gödel, 1932)
    % strongest intermediate logic (i.e. strengthening of intuitionistic logic) that is properly contained in classical logic

    \begin{itemize}\normalsize
    \item HT is based on Kripke semantics for intuitionistic logic
      \smallskip
    \item An HT model is a pair $(H,T)$ such that $H\subseteq T$
      \smallskip
    \item Implication is a genuine connective\only<2>{, and

      negation is defined in terms of implication: $\neg \varphi =  \varphi \to \bot$}
    \end{itemize}
  \item<4-> \structure{Discovery} (Pearce, 1996)% \uncover<7->{--- \alert<7->{Equilibrium logic}}\only<8>{ (\alert<8>{\textbf{non-monotonic}})}
    \medskip
    % Equilibrium models are special kinds of minimal N5 Kripke models

    Minimal HT models correspond to answer sets\pause[5], more precisely,

    an answer set $T$ of $\phi$ is
    \begin{itemize}\normalsize
    \item a total HT model $(T,T)$ of $\phi$ and
    \item there is no $H\subset T$ such that $(H,T)$ is an HT model of $\phi$
    \end{itemize}

  \item<6-> Such models are are called \alert<6->{equilibrium models} (or answer sets)
  \end{itemize}
\end{frame}
% ----------------------------------------------------------------------
\begin{frame}{Dynamic and Temporal Logic of Here-and-There}
  \begin{itemize}
  \item<2-> \structure{Structure} An HT trace is a sequence $(H_i,T_i)_{i=0}^{\lambda}$ of HT models
    \medskip
  \item<3-> \structure{Satisfaction} \ of a formula $\varphi$ in an HT trace \ $\M=(H_i,T_i)_{i=0}^{\lambda}$
    \smallskip
    \begin{itemize}\normalsize
    \item<4-> \structure{Something Boolean}
      \
      $\M, k \models \alert{\varphi \to \psi}$
      \pause[5]
      \small
      if
      \smallskip

      $( \mathbf{H}', \mathbf{T} ), k \not \models \varphi$
      or
      $( \mathbf{H}', \mathbf{T} ), k \models  \psi$ \ for all $\mathbf{H'} \in \{ \mathbf{H}, \mathbf{T} \}$
      \medskip

    \item<6-> \structure{Something Temporal}
      \small

      \smallskip
      $\M, k \models \alert{\alwaysF\varphi}$
      if
      $\M, i \models \varphi$ \ for all $i=k..\lambda$

      \smallskip
      $\M, k \models \alert{\eventuallyF\varphi}$
      if
      $\M, i \models \varphi$ \ for some $i=k..\lambda$
      \medskip

    \item<7-> \structure{Something Dynamic}
      \
      $\M, k \models \alert{[\rho]\varphi}$
      \pause[8]
      if\smallskip

      \small
      $( \mathbf{H'}, \mathbf{T} ),i \models \varphi$
      for all $i=0..\lambda$ with $(k,i) \in {\parallel\rho\parallel}^{( \mathbf{H'}, \mathbf{T} )}$ and $\mathbf{H'} \in \{ \mathbf{H}, \mathbf{T} \}$
    \end{itemize}
  \end{itemize}
\end{frame}
% ----------------------------------------------------------------------
\subsection{System}
% ----------------------------------------------------------------------
\begin{frame}[label=telingo,fragile]{\telingo}
  \begin{itemize}
  \item \structure{\telingo} % \only<2->{ \ --- \  \url{https://github.com/potassco/telingo}}
    \begin{itemize}\normalsize
    \item extends the full modeling language of \clingo{}\par with (past and future) temporal operators
    \item relies on finite trace
    \item implements an incremental translation
    \end{itemize}
%  \item \structure{\telingo} \ is an extension of \clingo{}
    \smallskip
  \item <2-> \structure{Primes} \ allow for expressing (iterated) next and previous operators
    \begin{itemize}
    \item $\bullet p(a)$ and $\circ q(b)$ can be expressed by \lstinline{'p(a)} and \lstinline{q'(b)}
    \end{itemize}
    \smallskip
  \item <3-> \structure{Example} \ \emph{``A robot cannot lift a box unless its capacity exceeds\par the box's weight plus that of all held objects''}
  \item <3-> []
\begin{lstlisting}[numbers=none,basicstyle=\ttfamily\small]
 :- lift(R,B), robot(R), capacity(R,C),
    #sum {   W :  box(B,W);
           V,O : 'holding(R,O,V) } > C.
\end{lstlisting}
  \end{itemize}
\end{frame}
% ----------------------------------------------------------------------
\begin{frame}[fragile,shrink]{Fox, Goose and Beans Puzzle}
\pause
\begin{lstlisting}[numbers=none,frame=none,basicstyle=\ttfamily]
#program always.
#(\pause#)
item(fox;beans;goose).
route(river_bank,far_bank). route(far_bank,river_bank).
eats(fox,goose). eats(goose,beans).
#(\pause#)
#program initial.

at(farmer,river_bank).
at(X,river_bank) :- item(X).
#(\pause#)
#program dynamic.

move(farmer).
0 { move(X) : item(X) } 1.

at(X,B) :- 'at(X,A), move(X), route(A,B).
:- move(X), item(X), 'at(farmer,A), not 'at(X,A).

at(X,A) :- 'at(X,A), not move(X).
#(\pause#)
#program always.

:- at(X,A), at(X,B), A<B.
:- eats(X,Y), at(X,A), at(Y,A), not at(farmer,A).

#program final.

:- at(X,river_bank).

#show move/1.
#show at/2.
\end{lstlisting}
\end{frame}
% ---------------------------------------------------------------------
\begin{frame}[fragile,shrink]{\telingo's solution}
\begin{lstlisting}[numbers=none,frame=none,basicstyle=\ttfamily]
UNIX > telingo river.tel
telingo version 1.0
Reading from river.tel
Solving...
Solving...
Solving...
Solving...
Solving...
Solving...
Solving...
Solving...
Answer: 1
 State 0:
  at(beans,river_bank) at(farmer,river_bank) at(fox,river_bank) at(goose,river_bank)
 State 1:
  move(farmer) move(goose)
  at(beans,river_bank) at(farmer,far_bank) at(fox,river_bank) at(goose,far_bank)
 State 2:
  move(farmer)
  at(beans,river_bank) at(farmer,river_bank) at(fox,river_bank) at(goose,far_bank)
 State 3:
  move(beans) move(farmer)
  at(beans,far_bank) at(farmer,far_bank) at(fox,river_bank) at(goose,far_bank)
 State 4:
  move(farmer) move(goose)
  at(beans,far_bank) at(farmer,river_bank) at(fox,river_bank) at(goose,river_bank)
 State 5:
  move(farmer) move(fox)
  at(beans,far_bank) at(farmer,far_bank) at(fox,far_bank) at(goose,river_bank)
 State 6:
  move(farmer)
  at(beans,far_bank) at(farmer,river_bank) at(fox,far_bank) at(goose,river_bank)
 State 7:
  move(farmer) move(goose)
  at(beans,far_bank) at(farmer,far_bank) at(fox,far_bank) at(goose,far_bank)
SATISFIABLE

Models       : 1+
Calls        : 8
Time         : 0.032s (Solving: 0.00s 1st Model: 0.00s Unsat: 0.00s)
CPU Time     : 0.028s
\end{lstlisting}
\end{frame}
% --------------------------------------------------------------------------------
\subsection{Application}
% --------------------------------------------------------------------------------
\begin{frame}[label=asprilo]{Robotic intra-logistics}
  \begin{columns}
    \begin{column}{0.69\textwidth}
      \begin{itemize}
      \item \alert{Robotics systems} for \alert{logistics} and \alert{warehouse automation}
        based on many
        \begin{itemize}
        \item mobile robots
        \item movable shelves
        \end{itemize}
        \smallskip
      \item Main tasks: \alert{order fulfillment}, i.e.
        \begin{itemize}
        \item routing
        \item order picking
        \item replenishment
        \end{itemize} % TODO: SPLIT HERE
        \smallskip
      \item Many competing \alert{industry solutions}:
        \begin{itemize}
        \item
          % CIM corp
          Amazon,
          Dematic,
          Genzebach,\par
          Gray Orange,
          Swisslog
        \end{itemize}
      \end{itemize}
    \end{column}
    \begin{column}{0.35\textwidth}  %%<--- here
      \begin{center}
        \includegraphics[width=\textwidth]{media/kiva1.png}
      \end{center}
    \end{column}
  \end{columns}
\end{frame}
% ----------------------------------------------------------------------
\begin{frame}{\asprilo's components}
  \bigskip
  \begin{itemize}
  \item<1-> \structure{Standardized benchmark domains}
  \item<1-> \structure{Formal specification}
  \item<1-> \structure{Versatile instance generator}
  \item<1-> \structure{Visualizer for problems and (candidate) solutions}
  \item<1-> \structure{Solution checker with error feedback}
    \bigskip
  \item<1->\structure{Reference ASP encodings}
  \end{itemize}
\end{frame}
% --------------------------------------------------------------------------------
\begin{frame}[c]{\asprilo's visualizer}
  \begin{center}
    \movie[label=asprilo-base,width=0.9\textwidth,height=0.72\textheight,poster,showcontrols]{}{media/asprilo-base.mp4}
  \end{center}
\end{frame}
%%% Local Variables:
%%% mode: latex
%%% TeX-master: "slides"
%%% End:
